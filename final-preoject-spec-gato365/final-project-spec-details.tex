% Options for packages loaded elsewhere
\PassOptionsToPackage{unicode}{hyperref}
\PassOptionsToPackage{hyphens}{url}
\PassOptionsToPackage{dvipsnames,svgnames,x11names}{xcolor}
%
\documentclass[
  letterpaper,
  DIV=11,
  numbers=noendperiod]{scrartcl}

\usepackage{amsmath,amssymb}
\usepackage{iftex}
\ifPDFTeX
  \usepackage[T1]{fontenc}
  \usepackage[utf8]{inputenc}
  \usepackage{textcomp} % provide euro and other symbols
\else % if luatex or xetex
  \usepackage{unicode-math}
  \defaultfontfeatures{Scale=MatchLowercase}
  \defaultfontfeatures[\rmfamily]{Ligatures=TeX,Scale=1}
\fi
\usepackage{lmodern}
\ifPDFTeX\else  
    % xetex/luatex font selection
\fi
% Use upquote if available, for straight quotes in verbatim environments
\IfFileExists{upquote.sty}{\usepackage{upquote}}{}
\IfFileExists{microtype.sty}{% use microtype if available
  \usepackage[]{microtype}
  \UseMicrotypeSet[protrusion]{basicmath} % disable protrusion for tt fonts
}{}
\makeatletter
\@ifundefined{KOMAClassName}{% if non-KOMA class
  \IfFileExists{parskip.sty}{%
    \usepackage{parskip}
  }{% else
    \setlength{\parindent}{0pt}
    \setlength{\parskip}{6pt plus 2pt minus 1pt}}
}{% if KOMA class
  \KOMAoptions{parskip=half}}
\makeatother
\usepackage{xcolor}
\setlength{\emergencystretch}{3em} % prevent overfull lines
\setcounter{secnumdepth}{-\maxdimen} % remove section numbering
% Make \paragraph and \subparagraph free-standing
\ifx\paragraph\undefined\else
  \let\oldparagraph\paragraph
  \renewcommand{\paragraph}[1]{\oldparagraph{#1}\mbox{}}
\fi
\ifx\subparagraph\undefined\else
  \let\oldsubparagraph\subparagraph
  \renewcommand{\subparagraph}[1]{\oldsubparagraph{#1}\mbox{}}
\fi


\providecommand{\tightlist}{%
  \setlength{\itemsep}{0pt}\setlength{\parskip}{0pt}}\usepackage{longtable,booktabs,array}
\usepackage{calc} % for calculating minipage widths
% Correct order of tables after \paragraph or \subparagraph
\usepackage{etoolbox}
\makeatletter
\patchcmd\longtable{\par}{\if@noskipsec\mbox{}\fi\par}{}{}
\makeatother
% Allow footnotes in longtable head/foot
\IfFileExists{footnotehyper.sty}{\usepackage{footnotehyper}}{\usepackage{footnote}}
\makesavenoteenv{longtable}
\usepackage{graphicx}
\makeatletter
\def\maxwidth{\ifdim\Gin@nat@width>\linewidth\linewidth\else\Gin@nat@width\fi}
\def\maxheight{\ifdim\Gin@nat@height>\textheight\textheight\else\Gin@nat@height\fi}
\makeatother
% Scale images if necessary, so that they will not overflow the page
% margins by default, and it is still possible to overwrite the defaults
% using explicit options in \includegraphics[width, height, ...]{}
\setkeys{Gin}{width=\maxwidth,height=\maxheight,keepaspectratio}
% Set default figure placement to htbp
\makeatletter
\def\fps@figure{htbp}
\makeatother

\KOMAoption{captions}{tableheading}
\makeatletter
\makeatother
\makeatletter
\makeatother
\makeatletter
\@ifpackageloaded{caption}{}{\usepackage{caption}}
\AtBeginDocument{%
\ifdefined\contentsname
  \renewcommand*\contentsname{Table of contents}
\else
  \newcommand\contentsname{Table of contents}
\fi
\ifdefined\listfigurename
  \renewcommand*\listfigurename{List of Figures}
\else
  \newcommand\listfigurename{List of Figures}
\fi
\ifdefined\listtablename
  \renewcommand*\listtablename{List of Tables}
\else
  \newcommand\listtablename{List of Tables}
\fi
\ifdefined\figurename
  \renewcommand*\figurename{Figure}
\else
  \newcommand\figurename{Figure}
\fi
\ifdefined\tablename
  \renewcommand*\tablename{Table}
\else
  \newcommand\tablename{Table}
\fi
}
\@ifpackageloaded{float}{}{\usepackage{float}}
\floatstyle{ruled}
\@ifundefined{c@chapter}{\newfloat{codelisting}{h}{lop}}{\newfloat{codelisting}{h}{lop}[chapter]}
\floatname{codelisting}{Listing}
\newcommand*\listoflistings{\listof{codelisting}{List of Listings}}
\makeatother
\makeatletter
\@ifpackageloaded{caption}{}{\usepackage{caption}}
\@ifpackageloaded{subcaption}{}{\usepackage{subcaption}}
\makeatother
\makeatletter
\@ifpackageloaded{tcolorbox}{}{\usepackage[skins,breakable]{tcolorbox}}
\makeatother
\makeatletter
\@ifundefined{shadecolor}{\definecolor{shadecolor}{rgb}{.97, .97, .97}}
\makeatother
\makeatletter
\makeatother
\makeatletter
\makeatother
\ifLuaTeX
  \usepackage{selnolig}  % disable illegal ligatures
\fi
\IfFileExists{bookmark.sty}{\usepackage{bookmark}}{\usepackage{hyperref}}
\IfFileExists{xurl.sty}{\usepackage{xurl}}{} % add URL line breaks if available
\urlstyle{same} % disable monospaced font for URLs
\hypersetup{
  pdftitle={Final Project Specifications},
  pdfauthor={Dr.~Williams},
  colorlinks=true,
  linkcolor={blue},
  filecolor={Maroon},
  citecolor={Blue},
  urlcolor={Blue},
  pdfcreator={LaTeX via pandoc}}

\title{Final Project Specifications}
\usepackage{etoolbox}
\makeatletter
\providecommand{\subtitle}[1]{% add subtitle to \maketitle
  \apptocmd{\@title}{\par {\large #1 \par}}{}{}
}
\makeatother
\subtitle{Stat 331/531}
\author{Dr.~Williams}
\date{}

\begin{document}
\maketitle
\ifdefined\Shaded\renewenvironment{Shaded}{\begin{tcolorbox}[boxrule=0pt, breakable, enhanced, interior hidden, frame hidden, sharp corners, borderline west={3pt}{0pt}{shadecolor}]}{\end{tcolorbox}}\fi

\hypertarget{final-project-specifications}{%
\subsubsection{Final Project
Specifications}\label{final-project-specifications}}

\hypertarget{overview}{%
\paragraph{\texorpdfstring{\textbf{Overview}}{Overview}}\label{overview}}

The final project for this course involves creating an R Shiny
application that analyzes data from one of four broad areas of
application: Spotify, Sports, Census Data, or Finance Data. This project
is designed to integrate concepts from each chapter of the textbook and
demonstrate practical applications using R. In your final project,
you'll also have the ability to incorporate ChatGPT, using it to make
your code, tables and visualizations within your R Shiny application
more presentable. This AI integration will add a cutting-edge dimension
to your project. Beyond this, the project will demonstrate your acquired
skills in R programming and data analysis, showcasing your ability to
apply classroom concepts to practical, real-world data scenarios.

\hypertarget{group-organization-and-selection-process}{%
\paragraph{\texorpdfstring{\textbf{Group Organization and Selection
Process}}{Group Organization and Selection Process}}\label{group-organization-and-selection-process}}

\begin{itemize}
\item
  \textbf{Total Number of Groups}: The class will be divided into 12
  groups, with each group consisting of exactly 3 members. This ensures
  a balanced team size for effective collaboration.
\item
  \textbf{Distribution Across Areas}: There will be three groups
  dedicated to each of the project areas. This structure allows for a
  focused exploration of each area while maintaining diversity in
  project topics.
\item
  \textbf{Group Formation}: Groups are self-selected based on mutual
  interest in a specific project area. This approach encourages students
  to collaborate with peers who share similar interests, fostering a
  more engaged learning environment.
\item
  \textbf{Fixed Group Membership}: Once formed, group membership is
  final. This policy is in place to encourage students to learn to work
  effectively with diverse personalities and perspectives, mirroring
  real-world team dynamics.
\item
  \textbf{Project Area Selection Process}:
\end{itemize}

The selection of project areas for the final project is an important
step in forming your groups and defining the focus of your work. To
ensure a fair and organized process, we will follow these steps:

\begin{enumerate}
\def\labelenumi{\arabic{enumi}.}
\tightlist
\item
  \textbf{Area Preference Survey (Week 2)}:

  \begin{itemize}
  \tightlist
  \item
    At the beginning of Week 2, all students will be required to
    complete a survey indicating their top two or three preferences for
    project areas, which include Spotify, Sports, Census Data, and
    Finance Data.
  \item
    The survey will also allow students to list one or two preferred
    teammates, if they have already formed a partial group.
  \item
    An additional question will ask students to prioritize between the
    project area or team formation, assisting in aligning student
    interests with available options.
  \end{itemize}
\item
  \textbf{Team and Area Assignment (End of Week 2)}:

  \begin{itemize}
  \tightlist
  \item
    Based on the survey responses, I will assign teams the project areas
    with the aim of ensuring that every student receives either their
    first or second choice.
  \item
    This approach will also support the formation of teams for students
    who are not yet part of a group, guaranteeing that everyone is
    included.
  \end{itemize}
\item
  \textbf{Confirmation and Initial Meeting}:

  \begin{itemize}
  \tightlist
  \item
    Following the assignment, students will be notified of their group
    and project area.
  \item
    Each group is then required to meet with me to discuss their project
    selection and initial ideas. This meeting should be scheduled by the
    end of Week 2.
  \end{itemize}
\end{enumerate}

This structured selection process, conducted during Week 2, ensures that
all students have an equal opportunity to work in their area of interest
and facilitates the formation of balanced teams. It is designed to cater
to both students with predetermined group preferences and those looking
for a group, ensuring a smooth start to the project phase of the course.

\hypertarget{week-project-timeline-with-detailed-tasks}{%
\subsubsection{10-Week Project Timeline with Detailed
Tasks}\label{week-project-timeline-with-detailed-tasks}}

\hypertarget{week-1-ice-breaker-and-team-formation}{%
\paragraph{Week 1: Ice Breaker and Team
Formation}\label{week-1-ice-breaker-and-team-formation}}

\begin{itemize}
\tightlist
\item
  \textbf{Task}: Conduct engaging activities to facilitate interaction.
\end{itemize}

\hypertarget{week-2-group-formation-and-area-selection}{%
\paragraph{Week 2: Group Formation and Area
Selection}\label{week-2-group-formation-and-area-selection}}

\begin{itemize}
\tightlist
\item
  \textbf{Task: Formation of Research Teams and Area Selection}

  \begin{itemize}
  \tightlist
  \item
    \textbf{Objective}: Assemble a team of three and choose a specific
    research area to delve into. Given the limited availability of spots
    per area, it's important to act swiftly in forming your group and
    making your selection. \textbf{Action Required}: Collaboratively
    decide on a project area that resonates with your group's interests
    and skills.
  \end{itemize}
\item
  \textbf{Submission Process: Preliminary Approval and Official
  Registration}

  \begin{itemize}
  \item
    \textbf{Official Registration}: Complete
    \href{https://forms.gle/FE64581hRqh8ppAMA}{Check-In 1} with your
    team details. Only one person has to complete this. This includes:
  \item
    \textbf{Team Members}: List all three group members.
  \item
    \textbf{Group Name}: Choose a creative name for your team.
  \item
    \textbf{Project Area}: Specify the area you plan to investigate.
    {[}Spotify, Sports, Census Data, or Finance Data{]}
  \item
    \textbf{Justification}: Briefly explain why you've chosen this
    particular area and what you hope to explore or achieve.
  \end{itemize}
\end{itemize}

\hypertarget{week-3-project-proposal-development}{%
\paragraph{Week 3: Project Proposal
Development}\label{week-3-project-proposal-development}}

\begin{itemize}
\item
  \textbf{You will be given your project area}
\item
  \textbf{Task}: Start crafting project proposals, focusing on
  objectives and data sources.
\item
  \textbf{Checkpoints}:

  \begin{itemize}
  \tightlist
  \item
    Assess whether project ideas align with course objectives. In other
    words, does the project cover one element from each chapter of the
    textbook? This hard to tell at this point but consult with me
    informally to make sure you are on the right track.
  \item
    You will most likely change your minds during week 6 and 7 but that
    is ok. This is just a starting point.
  \end{itemize}
\end{itemize}

\hypertarget{week-4-proposal-submission}{%
\paragraph{Week 4: Proposal
Submission}\label{week-4-proposal-submission}}

\begin{itemize}
\item
  \textbf{Task}: Submit project proposal detailing concept and data
  plans.
\item
  \textbf{Submission}: Complete
  \href{https://forms.gle/xiJNcRqwnwLD36qh8}{Check-In 2} with your team
  details. Only one person has to complete this. This includes:
\item
  Team Members
\item
  Group Name
\item
  Selected Area: {[} Spotify, Sports, Census Data, or Finance Data{]}
\item
  Potential Title of Your Project
\item
  What do you all hope to explore with this data? Why?
\end{itemize}

\hypertarget{week-5-data-acquisition-and-initial-development}{%
\paragraph{Week 5: Data Acquisition and Initial
Development}\label{week-5-data-acquisition-and-initial-development}}

\begin{itemize}
\tightlist
\item
  \textbf{Task}: Gain access to data and start developing initial code.
\item
  \textbf{Checkpoints}:

  \begin{itemize}
  \tightlist
  \item
    Ensure data is relevant and properly sourced.
  \item
    Check initial code for logical flow and functionality.
  \end{itemize}
\end{itemize}

\hypertarget{week-6-exploratory-data-analysis-and-discovery}{%
\paragraph{Week 6: Exploratory Data Analysis and
Discovery}\label{week-6-exploratory-data-analysis-and-discovery}}

\begin{itemize}
\tightlist
\item
  \textbf{Task}: Engage in exploratory data analysis to discover
  potential insights and directions. Students should:

  \begin{itemize}
  \tightlist
  \item
    \textbf{Experiment with Data}: Delve into the dataset to uncover
    interesting patterns, trends, or anomalies. This can involve
    statistical analysis, hypothesis testing, or trying out different
    data transformations.
  \item
    \textbf{Showcase Discoveries}: Present findings in an informal
    setting, highlighting any surprising insights or potential areas of
    further exploration.
  \item
    \textbf{Open Exploration}: Encourage creativity in exploring the
    data. This is the stage for brainstorming and playing around with
    different ideas without being committed to a final project
    direction.
  \end{itemize}
\end{itemize}

\hypertarget{week-7-refining-focus-for-r-shiny-application-and-data-preparation}{%
\paragraph{Week 7: Refining Focus for R Shiny Application and Data
Preparation}\label{week-7-refining-focus-for-r-shiny-application-and-data-preparation}}

\begin{itemize}
\tightlist
\item
  \textbf{Task}: This week, students will refine the focus of their R
  Shiny application using insights from their exploratory data analysis
  and start preparing their data for the application.

  \begin{itemize}
  \tightlist
  \item
    \textbf{Determine Application Focus}: Based on the exploratory
    analysis conducted in Week 6, decide on the specific aspect or
    question your R Shiny app will explore. This decision should
    leverage the data's potential to engage and inform users, focusing
    on a clear and achievable objective.
  \item
    \textbf{Data Preparation}: Once a direction has been selected,
    students are required to save their data from the API they are using
    as an RDA (R Data) file. This step ensures that the data is readily
    available in a format that can be efficiently used within the R
    environment, facilitating smoother development and deployment of the
    application.
  \item
    \textbf{Initial Visualization Concepts}: Begin brainstorming and
    sketching initial ideas for data visualization within the app. Focus
    on creating interactive, user-friendly, and informative
    visualizations that effectively communicate the data's story.
  \item
    \textbf{Flexibility to Pivot}: Remember, you have the option to
    adjust or change your project focus if a more compelling or feasible
    direction emerges from your data exploration. This flexibility is
    available until the end of Week 8, allowing for adaptive and
    responsive project development.
  \end{itemize}
\end{itemize}

\hypertarget{week-8-enhance-app-aesthetics-and-functionality}{%
\paragraph{Week 8: Enhance App Aesthetics and
Functionality}\label{week-8-enhance-app-aesthetics-and-functionality}}

\begin{itemize}
\tightlist
\item
  \textbf{Task}: Create app's UI/UX outline and develop script the video
  demonstration.
\item
  \textbf{Checkpoints}:

  \begin{itemize}
  \tightlist
  \item
    Assess the aesthetic appeal and user experience of the app.
  \item
    Review the script for clarity and comprehensiveness.
  \end{itemize}
\end{itemize}

\hypertarget{week-9-finalize-app-and-draft-report}{%
\paragraph{Week 9: Finalize App and Draft
Report}\label{week-9-finalize-app-and-draft-report}}

\begin{itemize}
\tightlist
\item
  \textbf{Task}: Complete app development and start the report draft.
\end{itemize}

\hypertarget{week-10-project-submission}{%
\paragraph{Week 10: Project
Submission}\label{week-10-project-submission}}

\begin{itemize}
\tightlist
\item
  \textbf{Task}: Discuss final version of the R Shiny app, video, and
  report with me
\item
  \textbf{Project Due March 19th at 11:59 PM during final exam week}
\end{itemize}

\hypertarget{examples-for-scenarios}{%
\paragraph{\texorpdfstring{\textbf{Examples for
Scenarios}}{Examples for Scenarios}}\label{examples-for-scenarios}}

\begin{enumerate}
\def\labelenumi{\arabic{enumi}.}
\item
  \textbf{Spotify App - Artist Rivalry Analysis}: Create an app that
  delves into the listening patterns related to recent music artist
  rivalries, like Billie Eilish vs.~Olivia Rodrigo. Analyze their
  streams, listener demographics, and popular playlists to understand
  fan engagement and preference trends.
\item
  \textbf{Sports Analytics App - NFL Rivalries}: Develop an application
  that focuses on key NFL rivalries, such as the San Francisco 49ers
  vs.~the Seattle Seahawks. Provide detailed statistics on team
  performance, player statistics, and historical matchup outcomes to
  offer insights for sports analysts and fans.
\item
  \textbf{Census Data Explorer - Urban Demographics Comparison}: Craft
  an interactive tool that visualizes and compares demographic data
  between major cities like San Francisco and Los Angeles. Analyze
  aspects like race demographics, age distribution, and population
  growth trends, drawing from recent census reports.
\item
  \textbf{Financial Trends Analyzer - Tech Industry Focus}: Build an app
  that tracks and predicts stock market trends, with a specific focus on
  the technology sector. Analyze stock performance, market news, and
  financial indicators to provide insights for investors and financial
  analysts.
\end{enumerate}

\hypertarget{project-requirements}{%
\paragraph{\texorpdfstring{\textbf{Project
Requirements}}{Project Requirements}}\label{project-requirements}}

\begin{enumerate}
\def\labelenumi{\arabic{enumi}.}
\tightlist
\item
  \textbf{Content Coverage}: The project must cover one element from
  each chapter of the textbook.
\item
  \textbf{R Shiny Application}: Students must use R Shiny, enabling
  users to interact with the data and insights provided by the app.
\item
  \textbf{Develop a Scenario}: Each group must develop a scenario or use
  case for their app regarding a project area, detailing what the app
  allows users to do, learn, or explore.
\item
  \textbf{Video Demonstration}: Create a 3-5 minute video demonstrating
  how the app is used and what users can gain from it.
\item
  \textbf{Report Submission}: Submit a 1-2 page report explaining the
  purpose of the app, its importance, a brief explanation of the code,
  and the role of ChatGPT in the project development.
\end{enumerate}

\hypertarget{final-project-deliverables}{%
\paragraph{Final Project
Deliverables}\label{final-project-deliverables}}

\begin{itemize}
\tightlist
\item
  \textbf{R Shiny Application}
\item
  \textbf{3-5 minute Video Demonstration}
\item
  \textbf{1-2 page Report Submission}
\item
  \textbf{Project Due March 19th at 11:59 PM}
\end{itemize}

\hypertarget{evaluation-criteria}{%
\paragraph{\texorpdfstring{\textbf{Evaluation
Criteria}}{Evaluation Criteria}}\label{evaluation-criteria}}

\begin{itemize}
\tightlist
\item
  Adherence to project requirements and timeline.
\item
  Creativity and practicality of the R Shiny application.
\item
  Quality of the video demonstration and report.
\item
  Effective use of R and integration of textbook concepts.
\end{itemize}



\end{document}
